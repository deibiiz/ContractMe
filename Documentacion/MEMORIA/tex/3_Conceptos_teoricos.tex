\capitulo{3}{Conceptos teóricos}

A continuación se sintetizarán algunos conceptos teóricos relevantes para la correcta compresión del proyecto.



\section{Introducción a la Blockchain}

La Blockchain proporcionando un registro inmutable y en tiempo real de transacciones, emergió como una solución al problema de doble gasto en transacciones digitales, un desafío que había eludido a los criptógrafos durante décadas.
En 2008 una persona o grupo anónimo bajo el seudónimo de Satoshi Nakamoto publicó un documento técnico para crear una moneda digital para contabilizar y transferir valor. Así nació una tecnología que se fundamenta en una base de datos distribuida compuesta por bloques de información replicados y sincronizados en múltiples ordenadores.  
Esta premisa hizo que en enero de 2009 entrará en funcionamiento la primera red basada en el protocolo Bitcoin.
La BlockChain demostró su capacidad para contabilizar y transferir valor de manera segura sin la necesidad de intermediarios, transformando el sector financiero y encontrando aplicaciones en una variedad de industrias.


\subsection{Tecnología subyacente}

La tecnología Blockchain funciona como un libro mayor digital (DTL) que registra transacciones en múltiples ordenadores de manera que cada registro es inalterable e irreversible. Este registro se organiza en bloques de datos que están interconectados de manera cronológica formando una cadena.

Cada bloque contiene un numero determinado de transacciones y esta formado por tres elementos principales; el dato de la transacción, el hash del bloque anterior en la cadena (lo que asegura la continuidad de la cadena) y su propio hash único, generado a partir de la información contenida en el bloque.
El hash es una función criptográfica que produce una salida de longitud fija a partir de una entrada de longitud variable, cualquier cambio en la información del bloque alteraría drásticamente su hash, evidenciando cualquier intento de fraude.

El proceso de añadir un nuevo bloque a la cadena requiere de un consenso entre los nodos de la red, lo cual se logra mediante diferentes mecanismos, siendo el "Proof of Work" (Prueba de trabajo) uno de los más utilizados. 
Este mecanismo implica resolver un problema criptográfico complejo que requiere un gran poder de computo. Funciona como una competición, en la que el nodo más rápido en resolver el problema obtiene el derecho de añadir el nuevo bloque a la cadena y es recompensado, generalmente con criptomonedas.

Una vez un bloque es añadido a la cadena, se distribuye a todos los nodos de la red, actualizando así el libro mayor en cada nodo. Esta distribución asegura una redundancia que la hace segura contra la manipulación, ya que alterar un registro requeriría cambiar el bloque correspondiente y todos los bloques posteriores en la mayoría de los nodos de la red, una tarea casi imposible debido a la demanda de computo.


\subsection{Utilidades de la Blockchain}

Su aplicación más reconocida ha sido Bitcoin, ha demostrado las capacidades de la Blockchain, permitiendo transacciones globales rápidas y seguras, caracterizadas por su alta liquidez, bajas comisiones y un nivel de anonimato que protege la privacidad del usuario.

Más allá de las criptomonedas, esta tecnología se destaca por su naturaleza descentralizada, donde cada individuo en la red tiene acceso a una copia del registro completo de transacciones, lo cual garantiza una transparencia sin precedentes. Siendo así que la Blockchain cuenta con la capacidad para ofrecer la trazabilidad completa en las cadenas de suministro. Cada producto puede ser rastreado desde su origen hasta el consumidor final, asegurando la autenticidad y facilitando la detección de cualquier problema en el proceso.

La seguridad es otro de los pilares fundamentales de la Blockchain, ya que la inmutabilidad del registro asegura que una vez la información ha sido añadida a la cadena, esta no podrá ser alterada, reforzando así la confianza en el sistema. 

Desde el punto de vista operativo, la Blockchain ofrece eficiencias significativas al eliminar los intermediarios, reduciendo tanto los tiempos de procesamiento como los costos asociados, a parte de minimizar las posibilidades de error.
Este aspecto es crucial en sectores como el financiero, donde los contratos inteligentes automatizan y aseguran la ejecución de acuerdos sin necesidad de intermediación.


\subsection{Tipos de Blockchain}

Existen diferentes tipos de redes, cada una diseña para satisfacer unas necesidades en cuanto a la privacidad, gobernanza y accesibilidad. 
Se pueden clasificar en cuatro categorías principales:

\begin{itemize}
\item \textbf{Blockchains públicas:} Son completamente abiertas y cualquier persona puede unirse. Bitcoin y Ethereum son buenos ejemplos de Blockchains públicas, donde las transacciones y los datos son visibles para todos, manteniendo al mismo tiempo el anonimato de los usuarios. Han pavimentando el camino para un ecosistema de aplicaciones descentralizadas (dApps) y finanzas descentralizadas (DeFi).

\item \textbf{Blockchains semiprivadas:} Son operadas por una única entidad con la posibilidad de restringir el acceso, ofreciendo un equilibrio entre el control y la descentralización. A diferencia de las redes privadas, las semiprivadas pueden permitir la participación de partes externas bajo ciertas condiciones, manteniendo un nivel significativo de control sobre la red.
Este tipo de redes las ofrecen empresas como IBM (Hyperledger Fabric) utilizada en sectores como la salud y la financiación, permite a las organizaciones configurar redes donde los datos se comparten solo con los actores autorizados, mejorando la eficiencia y la seguridad. Ofreciendo opciones personalizables para empresas, equilibrando la privacidad con la innovación.

\item \textbf{Blockchains privadas:} Son operadas por una única organización, permiten un control total sobre quién puede participar en la red. Estas redes limitan el principio de la descentralización pero ofrecen una solución eficaz para entornos empresariales que necesitan privacidad y eficiencia en procesos internos.
Por ejemplo, para este proyecto se ha utilizado la herramienta Ganache, la cual simula una red privada, siendo de gran utilidad en la fase de desarrollo y pruebas.

\item \textbf{Consorcio:} Representan un equilibrio entre los modelos públicos y privados, siendo operadas por un grupo de organizaciones en lugar de una única entidad. Esta posibilidad permite compartir la responsabilidad del mantenimiento de la red entre varios participantes, lo que las hace adecuadas para colaboraciones interempresariales. 
Un ejemplo real sería R3 Corda, que facilita transacciones eficientes y seguras entre instituciones financieras, reduciendo costos y tiempos de procesamiento.

\end{itemize} 



\section{Contratos inteligentes}

Un contrato inteligente es un código que ejecuta automáticamente los términos de un acuerdo entre partes. Los códigos, almacenados en la BlockChain, son ejecutados automáticamente, cuando se cumplen unas condiciones predefinidas, haciendo cumplir un acuerdo entre dos partes no confiables sin la necesidad de un tercero de confianza.

La idea fue conceptualizada por primera vez por Nick Szabo en 1993, visionando una nueva forma de establecer acuerdos digitales. Sin embargo, la falta de una plataforma adecuada mantuvo esta idea en teoría hasta la llegada de la Blockchain con Bitcoin en 2009, y mas notablemente con Ethereum en 2014, que los contratos inteligentes se materializaron prácticamente gracias a la infraestructura que esta tecnología proporciona.

Los contratos inteligentes utilizan la tecnología Blockchain para almacenar reglas, ejecutar automáticamente acciones cuando se cumplen esas reglas y almacenar los resultados en la blockchain. Debido a su naturaleza inmutable y distribuida, ofrecen un nivel de seguridad y confianza superior al de los sistemas tradicionales. Así mismo, al eliminar los intermediarios, ofrecen una reducción de costos y una mayor rapidez en la ejecución de acuerdos.


\section{Ethereum}


\section{Tokenización}


\section{Nodo}


\section{Wallet}


\section{Referencias}

Las referencias se incluyen en el texto usando cite~\cite{wiki:latex}. Para citar webs, artículos o libros~\cite{koza92}, si se desean citar más de uno en el mismo lugar~\cite{bortolot2005, koza92}.


\section{Imágenes}

Se pueden incluir imágenes con los comandos standard de \LaTeX, pero esta plantilla dispone de comandos propios como por ejemplo el siguiente:

\imagen{escudoInfor}{Autómata para una expresión vacía}{.5}



\section{Listas de items}

Existen tres posibilidades:

\begin{itemize}
	\item primer item.
	\item segundo item.
\end{itemize}

\begin{enumerate}
	\item primer item.
	\item segundo item.
\end{enumerate}

\begin{description}
	\item[Primer item] más información sobre el primer item.
	\item[Segundo item] más información sobre el segundo item.
\end{description}
	
\begin{itemize}
\item 
\end{itemize}

\section{Tablas}

Igualmente se pueden usar los comandos específicos de \LaTeX o bien usar alguno de los comandos de la plantilla.

\tablaSmall{Herramientas y tecnologías utilizadas en cada parte del proyecto}{l c c c c}{herramientasportipodeuso}
{ \multicolumn{1}{l}{Herramientas} & App AngularJS & API REST & BD & Memoria \\}{ 
HTML5 & X & & &\\
CSS3 & X & & &\\
BOOTSTRAP & X & & &\\
JavaScript & X & & &\\
AngularJS & X & & &\\
Bower & X & & &\\
PHP & & X & &\\
Karma + Jasmine & X & & &\\
Slim framework & & X & &\\
Idiorm & & X & &\\
Composer & & X & &\\
JSON & X & X & &\\
PhpStorm & X & X & &\\
MySQL & & & X &\\
PhpMyAdmin & & & X &\\
Git + BitBucket & X & X & X & X\\
Mik\TeX{} & & & & X\\
\TeX{}Maker & & & & X\\
Astah & & & & X\\
Balsamiq Mockups & X & & &\\
VersionOne & X & X & X & X\\
} 

Algunos conceptos teóricos de \LaTeX{} \footnote{Créditos a los proyectos de Álvaro López Cantero: Configurador de Presupuestos y Roberto Izquierdo Amo: PLQuiz}.