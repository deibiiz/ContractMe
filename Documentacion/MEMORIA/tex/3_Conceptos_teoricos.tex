\capitulo{3}{Conceptos teóricos}

A continuación se sintetizarán algunos conceptos teóricos relevantes para la correcta compresión del proyecto.



\section{Introducción a la Blockchain}

La Blockchain proporcionando un registro inmutable y en tiempo real de transacciones, emergió como una solución al problema de doble gasto en transacciones digitales, un desafío que había eludido a los criptógrafos durante décadas.
En 2008 una persona o grupo anónimo bajo el seudónimo de Satoshi Nakamoto publicó un documento técnico para crear una moneda digital para contabilizar y transferir valor. Así nació una tecnología que se fundamenta en una base de datos distribuida compuesta por bloques de información replicados y sincronizados en múltiples ordenadores.  
Esta premisa hizo que en enero de 2009 entrará en funcionamiento la primera red basada en el protocolo Bitcoin.
La BlockChain demostró su capacidad para contabilizar y transferir valor de manera segura sin la necesidad de intermediarios, transformando el sector financiero y encontrando aplicaciones en una variedad de industrias.


\subsection{Utilidades de la Blockchain}

Su aplicación más reconocida ha sido Bitcoin, la cual ha demostrado las capacidades de la Blockchain, permitiendo transacciones globales rápidas y seguras, caracterizadas por su alta liquidez, bajas comisiones y un nivel de anonimato que protege la privacidad del usuario.

Sin embargo, el alcance de la Blockchain va mucho más allá de las criptomonedas. Esta tecnología destaca por su naturaleza descentralizada, donde cada individuo en la red tiene acceso a una copia del registro completo de transacciones, lo cual garantiza una transparencia sin precedentes. Siendo así que la Blockchain cuenta con la capacidad para ofrecer la trazabilidad completa en las cadenas de suministro. Cada producto puede ser rastreado desde su origen hasta el consumidor final, asegurando la autenticidad y facilitando la detección de cualquier problema que pueda surgir a lo largo del camino.

Por otro lado, la seguridad es otro de los pilares fundamentales de la Blockchain, ya que la inmutabilidad del registro asegura que una vez la información ha sido añadida a la cadena, esta no podrá ser alterada. Esto se debe gracias a su estructura distribuida y al uso avanzado de algoritmos criptográficos logrando un entorno extremadamente seguro contra fraudes y ataques cibernéticos.

Desde el punto de vista operativo, la Blockchain ofrece eficiencias significativas al eliminar los intermediarios, reduciendo los tiempos de procesamiento y los costos asociados a parte de minimizar las posibilidades de error.


\subsection{Tipos de Blockchain}

Existen diferentes tipos de redes, cada una diseña para satisfacer unas necesidades en cuanto a la privacidad, gobernanza y accesibilidad. 
Se pueden clasificar en cuatro categorías principales:

\begin{itemize}
\item \textbf{Blockchains públicas:} Son completamente abiertas y cualquier persona puede unirse. Bitcoin y Ethereum son buenos ejemplos de Blockchains públicas, donde las transacciones y los datos son visibles para todos, manteniendo al mismo tiempo el anonimato de los usuarios.

\item \textbf{Blockchains semiprivadas:} Son operadas por una única entidad con la posibilidad de restringir el acceso, ofreciendo un equilibrio entre el control y la descentralización. A diferencia de las redes privadas, las semiprivadas pueden permitir la participación de partes externas bajo ciertas condiciones, manteniendo un nivel significativo de control sobre la red.
Este tipo de redes las ofrecen empresas como IBM (Hyperledger Fabric) con su plataforma de libro mayor. Tiene controles de privacidad avanzados, por lo que solo los datos que se desea compartir se comparten entre los participantes de la red "con permisos" conocidos.

\item \textbf{Blockchains privadas:} Son operadas por una única organización, permiten un control total sobre quién puede participar en la red. Estas redes limitan el principio de la descentralización pero ofrecen una solución eficaz para entornos empresariales que necesitan privacidad y eficiencia en procesos internos.
Por ejemplo, para este proyecto se ha utilizado la herramienta Ganache, la cual simula una red privada, siendo de gran utilidad en la fase de desarrollo y pruebas.

\item \textbf{Consorcio:} Representan un equilibrio entre los modelos públicos y privados, siendo operadas por un grupo de organizaciones en lugar de una única entidad. Esta posibilidad permite compartir la responsabilidad del mantenimiento de la red entre varios participantes, lo que las hace adecuadas para colaboraciones interempresariales. 
Un ejemplo real sería R3 Corda, una plataforma Blockchain de consorcio diseñada para la eficiencia en las transacciones financieras y otros sectores industriales.

\end{itemize} 


\section{Referencias}

Las referencias se incluyen en el texto usando cite~\cite{wiki:latex}. Para citar webs, artículos o libros~\cite{koza92}, si se desean citar más de uno en el mismo lugar~\cite{bortolot2005, koza92}.


\section{Imágenes}

Se pueden incluir imágenes con los comandos standard de \LaTeX, pero esta plantilla dispone de comandos propios como por ejemplo el siguiente:

\imagen{escudoInfor}{Autómata para una expresión vacía}{.5}



\section{Listas de items}

Existen tres posibilidades:

\begin{itemize}
	\item primer item.
	\item segundo item.
\end{itemize}

\begin{enumerate}
	\item primer item.
	\item segundo item.
\end{enumerate}

\begin{description}
	\item[Primer item] más información sobre el primer item.
	\item[Segundo item] más información sobre el segundo item.
\end{description}
	
\begin{itemize}
\item 
\end{itemize}

\section{Tablas}

Igualmente se pueden usar los comandos específicos de \LaTeX o bien usar alguno de los comandos de la plantilla.

\tablaSmall{Herramientas y tecnologías utilizadas en cada parte del proyecto}{l c c c c}{herramientasportipodeuso}
{ \multicolumn{1}{l}{Herramientas} & App AngularJS & API REST & BD & Memoria \\}{ 
HTML5 & X & & &\\
CSS3 & X & & &\\
BOOTSTRAP & X & & &\\
JavaScript & X & & &\\
AngularJS & X & & &\\
Bower & X & & &\\
PHP & & X & &\\
Karma + Jasmine & X & & &\\
Slim framework & & X & &\\
Idiorm & & X & &\\
Composer & & X & &\\
JSON & X & X & &\\
PhpStorm & X & X & &\\
MySQL & & & X &\\
PhpMyAdmin & & & X &\\
Git + BitBucket & X & X & X & X\\
Mik\TeX{} & & & & X\\
\TeX{}Maker & & & & X\\
Astah & & & & X\\
Balsamiq Mockups & X & & &\\
VersionOne & X & X & X & X\\
} 

Algunos conceptos teóricos de \LaTeX{} \footnote{Créditos a los proyectos de Álvaro López Cantero: Configurador de Presupuestos y Roberto Izquierdo Amo: PLQuiz}.