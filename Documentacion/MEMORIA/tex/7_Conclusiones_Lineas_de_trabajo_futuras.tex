\capitulo{7}{Conclusiones y Líneas de trabajo futuras}

Todo proyecto debe incluir las conclusiones que se derivan de su desarrollo. Éstas pueden ser de diferente índole, dependiendo de la tipología del proyecto, pero normalmente van a estar presentes un conjunto de conclusiones relacionadas con los resultados del proyecto y un conjunto de conclusiones técnicas. 
Además, resulta muy útil realizar un informe crítico indicando cómo se puede mejorar el proyecto, o cómo se puede continuar trabajando en la línea del proyecto realizado. 


\section{Conclusiones}

Tras la realización del proyecto, se logró desarrollar una aplicación móvil que implementa una solución basada en \textit{blockchain} y contratos inteligentes que aborda eficientemente los problemas de la economía sumergida.
Este proyecto ha sido una experiencia enriquecedora, permitiéndome familiarizarme con diversas tecnologías y herramientas que no había utilizado previamente.
Durante el desarrollo aprendí a programar Solidity y React Native y a usar Expo para el desarrollo de aplicaciones móviles. Además profundicé en el uso y funcionamiento con la \textit{blockchain} y mi familiaricé con herramientas como Firebase.

A lo largo del proyecto, me he encontrado diversas dificultades, muchas de las cuales se vieron acentuadas debido a mi falta de experiencia con algunas de las tecnologías. El uso de la \textit{blockchain} real presentó varios desafíos técnicos, especialmente en la integración con la aplicación Android. La conexión y uso de MetaMask que es relativamente sencillo en aplicaciones web, resultaron ser bastante más difícil en el entorno móvil.

A pesar de los desafíos y contratiempos, estoy satisfecho con el resultado final del proyecto. La experiencia fue extremadamente valiosa y me permitió aprender sobre múltiples tecnologías que tienen aplicaciones prácticas significativas.

\section{Lineas de trabajo futura}

El proyecto todavía tiene mucho potencial para crecer aún más, por lo que existen diversas mejoras que se podrían incorporar en futuros desarrollos. No cabe duda de que la principal línea futura de trabajo, contando con un pequeño presupuesto, debe ser desplegar el contrato en una red \textit{blockchain} real.
Implementar el contrato en una red real permitirá que la aplicación funcione en un entorno de producción, asegurando la conectividad entre la aplicación móvil y la \textit{blockchain}.
Del mismo modo, otro objetivo es lanzar la aplicación al mercado global, publicando la aplicación en tiendas de aplicaciones como Google Play, proporcionando mayor seguridad y confianza a los usuarios.

Aprovechando las funcionalidades que soportan los dispositivos móviles, sería interesante incorporar nuevas características como filtros de búsqueda avanzados y servicios de geolocalización.
Los filtros de búsqueda permitirían al usuario buscar contratos según diferentes criterios como ubicación, duración y salario, haciendo la búsqueda de contratos más eficiente.
Por otro lado, la implementación de geolocalización podría registrar automáticamente las horas de trabajo del empleado, asegurando el cumplimiento del contrato y facilitando la precisión en el pago del salario. Además, la geolocalización mejoraría la seguridad y transparencia del contrato, proporcionando un registro claro y verificable de las horas trabajadas.

