\apendice{Especificación de Requisitos}

\section{Introducción}

En esta sección se hace referencia a los requerimientos que debe cumplir el software para satisfacer las necesidades del cliente e identificar los componentes necesarios para entregar un producto adecuado. 


\section{Objetivos generales}

A lo largo de la memoria ya se han definido distintos objetivos del proyecto, pero estos pueden ser resumidos en los siguientes puntos:

\begin{itemize}

\item \textbf{Tecnología Blockchain:} Uso de la tecnología blockchain como base para la creación de un sistema de contratación descentralizado y seguro.
Se busca garantizar la transparencia e inmutabilidad de los datos, permitiendo que todas las transacciones queden registradas de forma segura.

\item \textbf{Contratos Inteligentes:} Desarrollo de contratos inteligentes que aseguraran el cumplimiento de acuerdos laborales.
Estos contratos deben de ser capaz de autoejecutarse en función de las condiciones preestablecidas por las partes. El principal objetivo es reducir la necesidad de intermediarios.

\item \textbf{Identificación segura mediante dispositivo móvil:} Implementación de métodos de autenticación y verificación seguros utilizando dispositivos móviles, como biometría o códigos QR.
Se pretende asegurar que solo los usuarios autorizados puedan acceder a un contrato y realizar operaciones dentro del sistema.

\item \textbf{Desarrollo de Aplicación Móvil:} Creación de una aplicación móvil android que sirva de interfaz para interactuar con los contratos inteligentes y la blockchain, escondiendo toda la complejidad al usuario.
La aplicación debe de estar diseñada para ser intuitiva, con el objetivo de que sea fácilmente utilizable para personas con cualquier nivel de habilidades tecnológicas.

\end{itemize}

\section{Catálogo de requisitos}

En este apartado se van a listar los requisitos específicos, agrupados en requisitos funcionales y no funcionales.


\subsection{Requisitos funcionales}

Este tipo de requerimientos están orientados a detallar las funciones que debe realizar la aplicación para cumplir con las expectativas del usuario. Por lo tanto se expondrá en como debe de ser el comportamiento del sistema en cada caso determinado.


Registro
Autenticación
Almacenamiento perfil
Verificación identidad
Modificación perfil
Información billetera

Creación contrato
Firma contrato (Códigos qr)
Visualización contrato
Modificación contrato
Administración contrato
Busqueda contrato

Almacenamiento contrato (no se si afecta al usuario)
Alertas
Muestra estadísticas


\subsection{Requisitos no funcionales}

Estos requisitos son utilizados para especificar criterios que puedan ser usados ara judgar la operación del sistema, más que detallar un comportamiento específico.

Seguridad
Disponibilidad
Escalabilidad
Rendimiento
Usabilidad
Mantenibilidad
Rendimiento



\section{Especificación de requisitos}





% Caso de Uso 1 -> Consultar Experimentos.
\begin{table}[p]
	\centering
	\begin{tabularx}{\linewidth}{ p{0.21\columnwidth} p{0.71\columnwidth} }
		\toprule
		\textbf{CU-1}    & \textbf{Ejemplo de caso de uso}\\
		\toprule
		\textbf{Versión}              & 1.0    \\
		\textbf{Autor}                & Alumno \\
		\textbf{Requisitos asociados} & RF-xx, RF-xx \\
		\textbf{Descripción}          & La descripción del CU \\
		\textbf{Precondición}         & Precondiciones (podría haber más de una) \\
		\textbf{Acciones}             &
		\begin{enumerate}
			\def\labelenumi{\arabic{enumi}.}
			\tightlist
			\item Pasos del CU
			\item Pasos del CU (añadir tantos como sean necesarios)
		\end{enumerate}\\
		\textbf{Postcondición}        & Postcondiciones (podría haber más de una) \\
		\textbf{Excepciones}          & Excepciones \\
		\textbf{Importancia}          & Alta o Media o Baja... \\
		\bottomrule
	\end{tabularx}
	\caption{CU-1 Nombre del caso de uso.}
\end{table}