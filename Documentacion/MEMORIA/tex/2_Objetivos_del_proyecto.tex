\capitulo{2}{Objetivos del proyecto}

En esta sección se muestran las metas que se pretenden alcanzar con el desarrollo del proyecto.

\section*{Objetivos técnicos}

\begin{enumerate}

\item \textbf{Integración de Ethereum}: Asegurando una implementación efectiva para el despliegue de contratos inteligentes

\item \textbf{Desarrollo de contratos inteligentes con Solidity}: Asegurando un código eficiente y adaptado a las necesidades específicas de la contratación laboral. 

\item \textbf{Desplegar en una red Ethereum real}: Logrando una convivencia con el resto de contratos de la comunidad.

\item \textbf{Implementación de tecnología biométrica}: Utilizando métodos como la huella dactilar o el reconocimiento facial para la identificación del usuario.

\item \textbf{Integración de tecnología GPS}: Verificando la presencia del trabajador en el lugar adecuado y el momento correcto.

\item \textbf{Desarrollo de una interfaz de usuario intuitiva y accesible}: Priorizando la simplicidad y la usabilidad para garantizar una plataforma fácil de utilizar para todos los usuarios.

\item \textbf{Evaluar la calidad de la solución}: En términos de eficiencia.

\item \textbf{Plan de implementeción y escalabilidad}

\end{enumerate}

%Había pensado en incluir integración de base de datos, inicio de sesión google y códigos qr

\section*{Objetivos del Software}

\begin{enumerate}

\item \textbf{Seguir estándares en el desarrollo de contratos inteligentes}: Alineándose con las mejores prácticas de la comunidad Ethereum y Solidity. 

\item \textbf{Adoptar metodologías ágiles}: Para favorecer un desarrollo iterativo y eficiente, facilitando la adaptación a cambios y mejora continua del proyecto.

\item \textbf{Documentación del proyecto}: Utilizando herramientas visuales para explicar la arquitectura. 

\item \textbf{Mantener un alto estándar de calidad en el código}: Empleando herramientas de revisión de código.



\end{enumerate}