\capitulo{2}{Objetivos del proyecto}

En esta sección se muestran las metas que se pretenden alcanzar con el desarrollo del proyecto.

\section*{Objetivos técnicos}

\begin{enumerate}

\item \textbf{Seguir estándares en el desarrollo de contratos inteligentes}: Alineándose con las mejores prácticas de la comunidad Ethereum y Solidity. 

\item \textbf{Adoptar metodologías ágiles}: Para favorecer un desarrollo iterativo y eficiente, facilitando la adaptación a cambios y mejora continua del proyecto.

\item \textbf{Documentar del proyecto}: Utilizando herramientas visuales para explicar la arquitectura. 

\item \textbf{Mantener un alto estándar de calidad en el código}: Empleando herramientas de revisión de código.


\item \textbf{Evaluar la calidad de la solución}: En términos de eficiencia.

\item \textbf{Planear un plan de implementeción y escalabilidad}

\end{enumerate}


\section*{Objetivos del Software}

\begin{enumerate}

\item \textbf{Integrar Ethereum en el proyecto}: Asegurando una implementación efectiva para el despliegue de contratos inteligentes

\item \textbf{Desarrollar contratos inteligentes con Solidity}: Asegurando un código eficiente y adaptado a las necesidades específicas de la contratación laboral. 

\item \textbf{Implementar tecnologías de autenticación biométrica}: Utilizando métodos como la huella dactilar o el reconocimiento facial para la identificación del usuario.

\item \textbf{implementar códigos QR}: Agilizando los procesos administrativos y operativos además de aplicar una capa adicional de seguridad.

\item \textbf{Recoger datos legales del usuario}: Integrando una base de datos donde se recoja datos personales del usuario para acreditarlo legalmente.

\item \textbf{Desarrollar una interfaz de usuario intuitiva y accesible}: Priorizando la simplicidad y la usabilidad para garantizar una plataforma fácil de utilizar para todos los usuarios.



\end{enumerate}