\capitulo{4}{Técnicas y herramientas}


\section{Herramientas}

Se muestra a continuación las herramientas usadas a lo largo del desarrollo del proyecto.

\subsection{React Native}

La elección del framework adecuado juega un papel crucial en el éxito de un proyecto. Dicha elección se complica aún más cuando se carece de experiencia previa en el desarrollo de aplicaciones móviles. Dentro del gran abanico de posibilidades React Native y Flutter emergen como grandes líderes en el sector gracias a sus grandes comunidades y la abundancia de recursos en línea. Por otro lado se pueden descartar directamente frameworks como Swift debido a su exclusividad para aplicaciones para iOS debido a que el objetivo de este proyecto es desarrollar una aplicación para Android.

React Native se presenta como mi elección favorita, este es un framework de código abierto creado por Facebook orientado a la creación de aplicaciones nativas tanto en iOS como en Android. React Native esta basado en JavaScript y React, una biblioteca de JavaScript destinada a la creación de interfaces de usuario. Fue lanzado en 2015 con el propósito de superar las limitaciones del desarrollo de aplicaciones móviles basado solo en HTML5, en las que simplemente se adapta aplicaciones web a un entorno móvil. React Native utiliza componentes nativos en lugar de WebViews para la interfaz de usuario, esto conlleva a que las aplicaciones se sientan y actúen como en una aplicación nativa. Uno de los elementos mas importantes de este framework es el "React Native Bridge", el cual facilita la comunicación entre el código JavaScript y los elementos nativos del dispositivos. El puente maneja de forma paralela dos flujos de trabajo, uno ejecuta la lógica de la aplicación en JavaScript y otro gestiona las operaciones de la interfaz de usuario nativa. Esto permite que las aplicaciones en React Native accedan a las características del dispositivo como la cámara o la ubicación manteniendo una experiencia fluida para el usuario. 

Mi preferencia de este framework sobre el resto radica en mi familiaridad previa con la programación web y JavaScript, lo cual reducirá la curva de aprendizaje en la transición hacia React Native en contraste con otros frameworks que podrían requerir el aprendizaje de nuevos lenguajes de programación o paradigmas de programación. Por otro lado, uno de los grandes motivos de este framework es su gran popularidad, el cual se traduce en un gran riqueza de recursos disponibles, como bibliotecas, videotutoriales o foros, crucial a la hora de enfrentar los desafíos que puedan surgir. Finalmente aunque para este proyecto no sea un requerimiento hay que tener en cuenta la gran versatilidad de React Native, el cual permite la creación de aplicaciones nativas tanto en Android como en iOS a partir de un único código base. Optimizando así el proceso de desarrollo permitiendo en un futuro poder dar cobertura a un mercado más amplio sin esfuerzos duplicados.

Como había nombrado anteriormente Flutter es otro framework que destaca sobre el resto y ofrece ciertas ventajas notables sobre Reac Native, como un rendimiento superior gracias al uso de su motor de renderizado propio y la gran capacidad de personalización de la interfaz de usuario. Aunque Flutter pudiera ser superior en algunos aspectos técnicos sigo decantándome por React Native debido a su gran comunidad y la familiaridad que tengo con las tecnologías web, priorizando así un aprendizaje más sencillo frente a posibles mejoras en el rendimiento. Por otro lado existen frameworks Ionic y Xamarin que los he descartado debido a sus limitaciones en términos de acceso a funciones nativas o una comunidad bastante inferior comparado con React Native o Flutter. Kotlin era un opción con gran potencial para el desarrollo nativo de aplicaciones Android pero la descarte rápidamente debido a que disponía de una curva de aprendizaje más pronunciada que su competencia y iba a representar un obstáculo significativo en cuanto al tiempo de desarrollo sin garantizar beneficios proporcionales a dicho esfuerzo.

Respecto a Angular, si bien este framework ofrece un ecosistema robusto para el desarrollo de aplicaciones web dinámicas y complejas utilizando TypeScript, es importante mencionar que también es posible desarrollar aplicaciones móviles con Angular, ofreciendo una experiencia cercana a la nativa, aunque a través de un enfoque diferente al de las aplicaciones nativas desarrolladas con React Native. Sin embargo, para el desarrollo específico de aplicaciones móviles nativas, React Native encaja mejor con los objetivos del proyecto.
Angular no deja de ser una elección excelente, pero debido a la preferencia del proyecto de enfocarse en una aplicación móvil sin la necesidad de dar soporte de navegador, Angular puede no ser la opción más adecuada para el proyecto.
Por lo tanto he descartado Angular buscando una solución más enfocada y optimizada para el desarrollo móvil, aprovechando las capacidades nativas de los dispositivos móviles.


\subsection{Expo}

Expo es un conjunto de herramientas, librerías y servicios para desarrollar aplicaciones nativas utilizando en Javascript y React Native. Proporciona un entorno de trabajo rico en funcionalidades que evita las complicaciones asociadas con la configuración nativa.
Expo ha sido fundamental en el proyecto para desarrollar la interfaz de usuario permitiendo probar y visualizar cambios en tiempo real en diferentes plataformas.
Una de las funcionalidades significativas de Expo ha sido su capacidad para facilitar la ejecución y testeo de la aplicación directamente en dispositivos móviles personales sin necesidad de emuladores. Esto se logra únicamente descargando la aplicación ExpoGo en el teléfono móvil y escaneando un código QR generado por el entorno de desarrollo Expo.
Expo tiene un papel fundamental en las fases de prueba y depuración, permitiendo probar la aplicación en un entorno real y ajustar la interfaz la interfaz de usuario con un ciclo de retroalimentación casi instantáneo.


\subsection{Solidity}

Solidity es un lenguaje de programación orientado a objetos de alto nivel diseñado para escribir Smart Contracts que se ejecutan en la Ethereum Virtual Machine (EVM). Solidity consta de una sintaxis similar a lenguajes como JavaScript, C++ y Python, lo que facilita su aprendizaje. Es un lenguaje de tipado estático, lo que significa que el tipo de cada variable se define y no cambia durante al ejecución del contrato.
Una de las potencialidades más destacadas de Solidity es su soporte para la herencia, una característica común en la programación orientada a objetos. Esto permite que los Smart Contracts hereden propiedades y comportamientos de otros contratos, beneficiándose de la reutilización de código y la organización de la lógica del mismo. En mi caso de gran utilidad para utilizar las funcionalidades del estándar ERC721 proporcionado por OpenZeppelin, que ofrece un conjunto de contratos inteligentes auditados y aprobados por su comunidad.


\subsection{Remix}

Remix es un entorno de desarrollo integrado (IDE) diseñado para el desarrollo de Smart Contracts escritos en Solidity. Proporciona una interfaz accesible y fácil de usar para escribir, compilar, probar y desplegar Smart Contracts directamente desde el navegador, sin la necesidad de instalar software adicional.
Remix también ofrece funcionalidades avanzadas como la compilación en tiempo real, el despliegue de contratos en diversas redes de prueba (testnets) y la interacción con SmartContracts ya desplegados.
Una de las características más útiles de este IDE es su análisis estático del código, pudiendo así identificar posibles errores de programación o vulnerabilidades de seguridad. Crucial para el desarrollo de Smart Contracts donde los errores pueden suponer consecuencias financieras significativas.
Además, Remix se integra con herramientas y plugins adicionales, ofreciendo un entorna mas rico y extenso permitiendo conectarse con herramientas y servicios como MetaMask, Truffle y Ganache ampliamente utilizados en mi proyecto.
Remix ha tenido una gran protagonismo en el desarrollo de mi proyecto permitiéndome iterar rápidamente a través de diferentes versiones de contratos inteligentes. Pudiendo ejecutar pruebas unitarias con diversas redes Ethereum como Rinkeby y Goerli, indispensable para validar la lógica del Smart Contract antes de su despliegue final. Dándome una gran capacidad de testeo en un entorno controlado pero realista el cual ha sido crucial para la corrección de errores y la optimización del uso de gas, asegurando así la eficiencia y seguridad de los contratos inteligentes desarrollados.


\subsection{Truffle}

Tras una fase inicial de desarrollo de los Smart Contracts usando Remix, la transición al uso de Truffle ha marcado un punto de inflexión en la complejidad de mi proyecto.
Truffle consiste en una suite de desarrollo avanzada para Ethereum, ofreciendo un conjunto de herramientas diseñadas para facilitar el desarrollo, gestión y despliegue de Smart Contracts.
Truffle proporciona un entorno de desarrollo estructurado generando una jerarquía de carpetas para favorecer la implementación de proyectos blockchain complejos.
Ahorra mucho tiempo con su sistema de migraciones y scripts de despliegue el cual automatiza y simplifica el proceso de lanzamiento de contratos.
Aunque Remix es una excelente herramienta, Truffle eleva la posibilidad de hacer pruebas unitarias a otro nivel con un marco de prueba mucho mas sofisticado, permitiendo ejecutar test en Solidity o JavaScript.
Finalmente, uno de las mayores ventajas de usar Truffle en el desarrollo de aplicaciones descentralizadas es en cómo simplifica el proceso de unir el trabajo que se realiza en el backend con el frontend. Cuenta con herramientas que ayudan a conectar Smart Contracts con aplicaciones móviles haciendo que la conexión sea mas directa y menos propensa a errores.


\subsection{OpenZeppelin}

OpenZeppelin es una biblioteca para el desarrollo seguro de Smart Contracts en Ethereum. Ofrece implementaciones auditadas y probadas para minimizar riesgos de seguridad.
Su uso ha sido crucial para el desarrollo de mi proyecto, usando el estándar ERC721 para la representación de los contratos como tokens no fungibles (NFTs) asegurando que la aplicación cumpla con los estándares de seguridad.


\subsection{Ganache}

Ganache funciona como un nodo de Ethereum personal, permitiendo a los desarrolladores simular un entorno de blockchain que opera localmente ofreciendo un espacio seguro y controlado para experimentar sin ningún costo real ni tiempos de espera con las redes públicas de Ethereum.
Proporciona diez cuentas cargadas con 1000 ETH cada una y con su clave pública y privada correspondiente.
Ganache se puede usar tanto en la línea de comandos como en en la aplicación de escritorio y proporciona  una vista de las transacciones, bloques y estado de la red, mostrando el feedback en tiempo real, vital para realizar una iteración rápida.
Ganache se integra perfectamente con Truffle, siendo Ganache el entorno de desarrollo local predeterminado de Truffle.


\subsection{MetaMask}

MetaMask es una extensión de navegador y una aplicación móvil que permite a los usuarios interactuar con la blockchain de manera segura y sencilla. Actúa como puente entre los navegadores web y la blockchain.
Su funcionamiento es análogo a una cartera digital, permitiendo a los usuarios almacenar sus cuentas de Ethereum.
Al conectarse a dApps, los usuarios pueden usar sus cuentas de MetaMask para autenticarse eliminando la necesidad de copiar claves privadas manualmente. Por otro lado, aplica una capa adicional de seguridad ya que esta herramienta encripta la información del usuario y almacena las claves privadas directamente en el dispositivo del usuario. Esto asegura que solo el usuario tenga acceso a sus fondos y datos.
La integración de MetaMask en mi proyecto no solo mejora la experiencia del usuario final, sino que también agiliza el desarrollo de mi proyecto al proporcionar una manera fácil y segura de acceder a los activos de los usuarios.


\subsection{Node.js}

Node.js es un entorno de ejecución en JavaScript, tradicionalmente un lenguaje de programación del lado del cliente, para desarrollar aplicaciones del lado del servidor. Es conocido por su capacidad para manejar operaciones asíncronas y por su escalabilidad siendo de gran popularidad en el desarrollo de aplicaciones web modernas. Una de las ventajas de de Node.js es su ecosistema de paquetes gestionados por npm (Node Package Manager), que proporciona acceso a miles de librerías.
En el proyecto se ha usado de manera frecuente, más allá de ser un requisito para utilizar Truffle, a través del uso de npm install se ha usado "npm install" para descargar librerías y paquetes necesarios para el desarrollo del frontend y backend.
Bien es así, que aunque React Native sea el framework principal del desarrollo del frontend, la gestión de sus dependencias, librerías adicionales se realiza mediante npm, el cual opera sobre Node.js.
A su vez, por ejemplo el uso de la biblioteca Web3.js, una de las mas importantes del proyecto ya que es fundamental para interactuar con la blockchain, se ha instalado y gestionado usando npm.
Por tanto, Node.js ha sido una pieza crucial en el desarrollo del proyecto, simplificando la configuración del proyecto.


