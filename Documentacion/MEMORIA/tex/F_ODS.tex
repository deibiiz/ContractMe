\apendice{Anexo de sostenibilización curricular}

\section{Introducción}

La transformación digital está remodelando diversos sectores económicos. Sin embargo, algunos sectores como la agricultura, los servicios domésticos y el trabajo freelance enfrentan desafíos significativos en el proceso de contratación y verificación de identidad.
En esta sección, se propone una solución alineada con las directrices de sostenibilidad en el curriculum universitario aprobadas por la CRUE.


\subsection{Sostenibilidad social y económico}

la sostenibilidad social implica la creación de un entorno laboral justo y equitativo.
La utilización de blockchain y contratos inteligentes asegura la transparencia y seguridad en los contratos laborales, garantizando que los términos acordados se cumplan automáticamente y sin intervención manual. 
Este enfoque combate con la informalidad laboral y protege al trabajador, asegurando que se cumplan unas condiciones laborales dignas, en línea con el principio ético de las directrices de sostenibilidad. 

Desde una perspectiva económica, fomentar técnicas legales y justas que aseguren la reducción de la economía sumergida, fortalece la economía formal y aumenta la recaudación fiscal.
Al implementar contratos inteligentes, la aplicación provee un registro inmutable de todas las transacciones, facilitando la fiscalización y el cumplimiento de las normativas.
Esto incrementa la confianza en el sistema laboral y fomenta un desarrollo económico sostenible.


\subsection{Formación para la sostenibilidad}

En el proyecto también se reconoce y se valora la importancia de la educación en sostenibilidad. La formación de los usuarios en el uso de la blockchain y contratos inteligentes refuerza la capacidad de tomar decisiones informadas y responsable, fomentando la cultura de transparencia y equidad en el mercado laboral perseguida con el desarrollo de dicho proyecto.
Esta formación es clave para desarrollar competencias transversales en sostenibilidad.


\subsection{Accesibilidad}

La accesibilidad es crucial en el desarrollo de nuestro proyecto, especialmente enfocado para los trabajadores de sectores marginados, permitiendo el acceso a unas condiciones laborales justas y transparentes.
La aplicación está diseñada para ser simple y accesible desde dispositivos móviles, capaz de ejecutarse en
incluso en dispositivos con muy pocos recursos.
La integración de tecnologías de autenticación biométrica proporciona una capa adicional de seguridad, garantizando la privacidad y protección de la identidad de los trabajadores. 


\subsection{Impacto Medioambiental}

Aunque el foco principal del proyecto se encuentra en la sostenibilidad social y económica, también se considera el impacto medioambiental de la solución. 
La digitalización de procesos laborales reduce la dependencia del papel y disminuye la necesidad de
desplazamientos físicos, contribuyendo a la reducción de la huella de carbono. La eficiencia y automatización proporcionadas por las tecnologías empleadas optimizan el uso de recursos, minimizando el desperdicio y promoviendo prácticas más sostenibles.

\section{Conclusión}

La integración de sostenibilidad en el proyecto presentado aborda de manera integral los desafíos de la economía sumergida y promueve prácticas laborales justas y transparentes. Al adoptar tecnologías disruptivas como blockchain y contratos inteligentes, garantizamos la formalización del empleo, la protección de los derechos de los trabajadores y el desarrollo económico sostenible. 
Este enfoque holístico y transversal busca que no solo mejore la eficiencia operativa sino que también tenga un impacto positivo duradero en la sociedad. Al integrar principios de sostenibilidad en el diseño, promovemos un entorno laboral más justo, seguro y equitativo, alineado con los objetivos de desarrollo sostenible.