\capitulo{1}{Introducción}

La transformación digital se ha convertido en un pilar fundamental para el desarrollo y la eficiencia de diversos sectores económicos, entre ellos el mercado laboral. Sin embargo, a pesar de los avances tecnológicos, ciertos sectores como la agricultura, los servicios domésticos y el trabajo \textit{freelance} enfrentan retos significativos en la gestión eficiente de contrataciones y la verificación de identidad.
Estos desafíos se ven agravados por la presencia de la economía sumergida, donde la falta de transparencia y la informalidad laboral no solo perjudican la economía global, sino que también vulneran muchos de los derechos de los trabajadores.
En este contexto, se pretende implementar una solución innovadora que permita superar las barreras existentes, haciendo frente a la economía sumergida mediante el empleo de tecnologías disruptivas como \textit{blockchain} y contratos inteligentes. Esta iniciativa busca promover la formalización de empleos y asegurar prácticas laborales justas, abordando directamente los problemas de falta de transparencia y seguridad en los procesos contractuales. Se destacará el impacto social y económico de reducir la economía sumergida, apoyándose en datos que evidencian su magnitud y las implicaciones positivas de una mayor formalización laboral.

La adopción de contratos inteligentes en una red \textit{blockchain} permite una gestión contractual transparente, segura y automática, asegurando que los términos acordados se cumplan eficazmente sin intervención manual.
Esta tecnología también proporciona un registro inmutable y verificable de las transacciones y acuerdos labores, combatiendo así prácticas ilegales y fomentando un entorno de trabajo más equitativo.

Este proyecto no solo aborda la necesidad de un sistema de contratación y verificación más eficiente y seguro sino que también se anticipa a las demandas de un mercado laboral en constante evolución, ofreciendo una herramienta adaptable y escalable. Con una interfaz de usuario diseñada para la simplicidad y la accesibilidad en dispositivos móviles, se busca fomentar prácticas de empleo legales y transparentes, contribuyendo a la creación de un entorno laboral más justo y seguro para todos.
La inclusión de tecnologías de autenticación biométrica refuerza este objetivo, asegurando la identidad de los trabajadores de manera segura y privada.

Por tanto este proyecto no solo propone una solución práctica que beneficia tanto a trabajadores como empleadores, sino que también tiene el potencial de impactar positivamente en la economía global, marcando un paso adelante hacia la erradicación de la economía sumergida y el establecimiento de prácticas laborales más justas y transparentes a nivel mundial.