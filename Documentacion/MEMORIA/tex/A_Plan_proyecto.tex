\apendice{Plan de Proyecto Software}


\section{Introducción}

En este apartado se recoge el ciclo de vida del proyecto, detallando los aspectos más relevantes del mismo y como se han resuelto los diferentes problemas a lo largo de su desarrollo. Se presentarán secciones que muestran de manera cronológica la justificación de las decisiones tomadas.

Para llevar a cabo este seguimiento y planificación del proyecto se ha utilizado una metodología Scrum, que ha permitido un desarrollo ágil dividido en sprints de dos semanas cada uno. 
Al comienzo de cada sprint se establecen las tareas y objetivos a realizar durante ese periodo. Al final de cada sprint se realizan reuniones con los tutores para valorar los resultados obtenidos y definir nuevas tareas para el siguiente sprint.
Para organizar las diferentes tareas se ha usado Gitlab que permite visualizar los diferentes estados de desarrollo de las tareas.


\section{Planificación temporal}

La propuesta del proyecto consistía en crear una aplicación android basada en blockchain que simplifica la contratación y la verificación a través de contratos inteligentes.
Los requisitos principales del proyecto se pueden dividir en los siguiente puntos:

\begin{itemize}

\item \textbf{Tecnología Blockchain:} Utilizar la tecnología blockchain para desplegar contratos inteligentes que gestionen automáticamente los contratos laborales, desde su creación hasta su ejecución.

\item \textbf{Contratos Inteligentes:} Implementar contratos inteligentes en Python, Vyper y Solidity.

\item \textbf{Localización GPS:} Integrar tecnología GPS para permitir a los empleadores imponer zonas de trabajo específicas.

\item \textbf{Identificación segura mediante dispositivo móvil:} Implementar autenticación biométrica y el escaneo de códigos QR.

\item \textbf{Integración con Pagos:} Incorporar procesamiento de pagos dentro de la aplicación para facilitar transacciones rápidas y seguras

\item \textbf{Desarrollo de Aplicación Móvil:} Diseñar una interfaz de usuario amigable para dispositivos móviles que facilite la creación de contratos, seguimiento y pago.

\end{itemize}

Una vez expuestos los requerimientos principales del proyecto, la etapa inicial del proyecto se basó en una exhaustiva investigación que permitió obtener un conocimiento detallado sobre las tecnologías y herramientas necesarias. Dado que inicialmente no contaba con conocimientos previos en desarrollo de aplicaciones móviles y tecnología blockchain, esta investigación fue crucial para identificar las mejores prácticas y soluciones en estos campos. 
Esta etapa de investigación y adaptación a las nuevas tecnologías tuvo una duración de aproximadamente 3 semanas, durante el cual se hizo especial énfasis en entender a fondo el funcionamiento de la blockchain y los contratos inteligentes.
Este aprendizaje teórico se reforzó de manera practica con diversos proyectos usando Truffle, completando videoTutoriales y realizando el curso interactivo CryptoZombies, además de consultar numerosos artículos especializados. Estas actividades facilitaron la asimilación del nuevo lenguaje de programación y familiarización con el entorno blockchain.


\subsubsection{Sprint 0}

Este sprint se desarrolló entre los días 3 y 17 de Noviembre de 2023. Se realizaron las siguientes tareas y objetivos:

\begin{enumerate}

\item \textbf{Configuración repositorio:}

\item \textbf{Configuración entorno para la redacción de la memoria:}

\item \textbf{Investigación de tecnologías para realizar una app móvil:}

\item \textbf{Diseñar la arquitectura del proyecto:}

\end{enumerate}




\section{Estudio de viabilidad}

\subsection{Viabilidad económica}

\subsection{Viabilidad legal}


