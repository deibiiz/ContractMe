\apendice{Documentación técnica de programación}

\section{Introducción}

En esta sección se describe la estructura del proyecto, el proceso de instalación y las herramientas necesarias para desarrollar el trabajo. También se explica cómo realizar la instalación de dependencias, la compilación, la ejecución del proyecto y el despliegue en Expo.

\section{Estructura de directorios}

En el primer nivel, se encuentran dos directorios, \textbf{Documentacion} y \textbf{Code}. El primero de ellos recoge los archivos de documentación sobre el proyecto, como la memoria y sus anexos, por otra parte, el segundo recoge la lógica de la aplicación.

Dentro de Code, el directorio \textbf{AppContractMe} contiene todo el código fuente necesario para la interfaz de usuario de la aplicación. Se divide en varios subdirectorios que organizan los recursos y scripts de manera lógica, siendo \textbf{scr} el más importante, incluyendo:

\begin{itemize}

\item \textbf{AppLogin:} Contiene los componentes para la autenticación y gestión de sesiones de usuario.

\item \textbf{ContractConexion:} Incluye las funciones que facilitan la conexión y la interacción con los contratos inteligentes.

\item \textbf{Screens:} Agrupa las diferentes pantallas de la aplicación, permitiendo la navegación al usuario.

\item \textbf{components:} Reúne elementos reutilizables que se emplean en diversas partes de la aplicación, como botones y ciertas funcionalidades.

\end{itemize}

Por otro lado, también existe el directorio \textbf{SmartContract} que recoge toda la lógica del contrato inteligente e incluye:

\begin{itemize}

\item \textbf{build:} Contiene los archivos compilados de los contratos, que son necesarios para su despliegue y ejecución en la blockchain.

\item \textbf{constracts:} Alberga los scripts de los contratos inteligentes.

\item \textbf{migrations:} Gestiona el scripts que ayuda en la migración y despliegue de los contratos en la blockchain.

\item \textbf{node\_modules:} Directorio que incluye las dependencias de Node.js utilizadas en el proyecto.

\item \textbf{test:} Contiene los tests escritos en java para asegurar el correcto funcionamiento de los contratos inteligentes.

\end{itemize}

Finalmente hay otro directorio llamada \textbf{ganache\_db} el cual se utiliza para almacenar la configuración y los datos de la base de datos de Ganache,


\section{Manual del programador}

Este apartado servirá de ayuda y referencia a futuros desarrolladores que quieran replicar el proyecto. 
Por ello se detallará los requisitos necesarios y el proceso de instalación para el desarrollo de la aplicación. 

\subsection{Entorno desarrollo}

Antes de explicar la instalación de los programas y herramientas para el desarrollo de la aplicación es necesario detallar unas especificaciones mínimas en cuanto al equipo de desarrollo para poder trabajar con el proyecto.

\begin{enumerate}

\item\textbf{Procesador (CPU):}
	\begin{itemize}
	\item\textbf{Mínimo:} Procesador con arquitectura x86\_64, 2º generación de Intel o procesador AMD que
	 soporte \"Hypervisor FrameWork\", Permitiendo una compilación rápida de código y emular un dispositivo 	
	 móvil para pruebas.
	\item\textbf{Recomendado:} Intel Core i7 o AMD ryzen 7, con 6-8 núcleos.
	\end{itemize}
	
\item\textbf{Memoria RAM:}
	\begin{itemize}
	\item\textbf{Mínimo:} 8 GB de RAM como mínimo, para manejar diversas tareas y la ejecución de
	simuladores.
	\item\textbf{Recomendado:} 16 GB de RAM o más, pudiendo mantener múltiples dispositivos emulados
	simultáneamente.
	\end{itemize}
	
\item\textbf{Sistema operativo:}
	\begin{itemize}
	\item\textbf{Mínimo:} Windows 10 64-bit.
	\item\textbf{Recomendado:} Windows 11.
	\end{itemize}

\end{enumerate}

Por otro lado, aunque no es un requisito obligatorio, se recomienda contar con un dispositivo Android físico para ejecutar la aplicación y hacer uso de funcionalidades nativas como el reconocimiento de huella, siendo necesario las siguientes especificaciones.

\begin{enumerate}

\item\textbf{Versión Android:}
	\begin{itemize}
	\item\textbf{Mínimo:}  Android 6.0 (Marshmallow)
	\item\textbf{Recomendado:} Android 9.0 (Pie) o superior.
	\end{itemize}
	
\item\textbf{Procesador (CPU):}
	\begin{itemize}
	\item\textbf{Mínimo:} Quad-core 1.2 GHz.
	\item\textbf{Recomendado:} Octa-core 2.0 GHz o superior.
	\end{itemize}
	
\item\textbf{RAM:}
	\begin{itemize}
	\item\textbf{Mínimo:} 2 GB de RAM.
	\item\textbf{Recomendado:} 4-8 GB de RAM.
	\end{itemize}
	
\item\textbf{Compatibilidad con características nativas:}
	\begin{itemize}
	\item\textbf{Huella digital:} El dispositivo debe contar con un sensor de huellas dactilares compatible con
	las API de Android para autenticación biométrica.
	\item\textbf{Reconocimiento facial:} cámara interior con capacidad de reconocimiento facial.
	\item\textbf{Cámara trasera:} Cámara de al menos 8 MP para la lectura de códigos QR.
	\end{itemize}

\end{enumerate}


\section{Compilación, instalación y ejecución del proyecto}

\subsection{Instalación}

Seguidamente se detallará como configurar un entorno de desarrollo para poder trabajar en la aplicación \"ContractMe\".

\begin{enumerate}

\item \textbf{Node.js:} Node.js es una plataforma de ejecución para JavaScript del lado del servidor y es esencial para la gestión de paquetes y la ejecución de varias herramientas de desarrollo. 
Para instalar Node.js es necesario dirigirse a la \href{https://nodejs.org/en}{pagina oficial de Node.js}.y seleccionar el instalador para Windows. Ver imagen \ref{fig:InstalarNodejs}.

Al ejecutar el instalador que se ha descargado, es necesario seleccionar la opción que permite instalar npm y añadir Node.js a tu path. Ver imagen \ref{fig:NodejsSetUp}.  
En concreto la versión utilizada de Node.js para el desarrollo del proyecto ha sido la v18.18.2 y la versión utilizada de npm ha sido la 10.2.2.

\imagen{InstalarNodejs}{Descarga de Node.js}{1}
\imagen{NodejsSetUp}{Instalar Node.js y npm}{1}

\item \textbf{Truffle Suite}: Teniendo Node.js y npm instalados, procederemos a instalar Truffle globalmente, para ello desde la terminal ejecutaremos el siguiente comando: \texttt{npm install -g truffle} lo cual nos permitirá usar el comando 

\item \textbf{React Native CLI}

\item \textbf{Expo CLI}



\item \textbf{Ganache}

\end{enumerate}

\section{Pruebas del sistema}
